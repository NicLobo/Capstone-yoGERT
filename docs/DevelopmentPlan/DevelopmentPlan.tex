\documentclass{article}

\usepackage{booktabs}
\usepackage{tabularx}
\usepackage{graphicx}
\usepackage{float}


\date{September 25, 2022}

%\input{../Comments}
%%% Common Parts

\newcommand{\progname}{ProgName} % PUT YOUR PROGRAM NAME HERE
\newcommand{\authname}{Team \#, Team Name
\\ Student 1 name and macid
\\ Student 2 name and macid
\\ Student 3 name and macid
\\ Student 4 name and macid} % AUTHOR NAMES                  

\usepackage{hyperref}
    \hypersetup{colorlinks=true, linkcolor=blue, citecolor=blue, filecolor=blue,
                urlcolor=blue, unicode=false}
    \urlstyle{same}
                                


\begin{document}

\title{Development Plan\\\prognam}

\author{Smita Singh, Abeer Alyasiri, Niyatha Rangarajan,\\ Moksha Srinivasan, Nicholas Lobo, Longwei Ye}

\begin{table}[hp]
\caption{Revision History} \label{TblRevisionHistory}
\begin{tabularx}{\textwidth}{llX}
\toprule
\textbf{Date} & \textbf{Developer(s)} & \textbf{Change}\\
\midrule
Sept. 21, 2022 & Abeer & Introduction, Team Members Roles, Proof of Concept Demonstration, Project Scheduling, Coding Standard \\
Sept. 21, 2022 & Smita & Worked on Team Meeting Plan, Team Communication Plan, Workflow Plan, Technology, Coding Standard\\
%... & ... & ...\\
\bottomrule
\end{tabularx}
\end{table}

\newpage

\maketitle

This document will explore the development plan of Capstone project in terms of team logistics, project design, testing plan, and project execution %A

\section{Team Meeting Plan}
The plan is to hold meetings every Monday evening from 6:30 to 7:30pm. Meetings will be conducted through Microsoft Teams or when necessary will be held in-person at McMaster's Hatch building. A meeting agenda will be established before each meeting in order to keep the meeting succinct and on track. The meetings will be used for division of labour, and completing deliverables. %S
Other meetings will be held to conduct group work sessions to complete assigned tasks. These meetings will be less formal and will be scheduled based on the project scheduling.

\section{Team Communication Plan}
The primary form of communication will be through Microsoft teams, but a secondary form will be discord for urgent communication. Scheduling meetings and distributing tasks will be communicated through a shared Microsoft Teams channel. %S

\section{Team Member Roles} 
All team members are expected to be responsible for completing technical software component to the project. The default roles assigned for the moment is to divide general topics across the project for members to have more knowledge than others. 
\begin{table}[H]
    \centering
    \begin{tabular}{|c|c|p{60mm}|}
         \hline
         Name & Primary Role & Description\\
         \hline
         Moksha Srinivasan & Team Leader & Primary Ticket Creator, Developer\\
         \hline
         Abeer Alyasiri & Scribe & Liason between TA and supervisor, Developer\\
         \hline
         Nicholas Lobo & Git Expert & Managing and resolving git conflicts, Developer \\
         \hline
         Niyatha Rangarajan & Documentation Expert & Responsible for managing documentation, Developer \\
         \hline
         Longwei Ye & Technology Expert & Responsible for gaining knowledge on new technologies, Developer  \\
         \hline
         Smita Singh & LaTeX Expert & Responsible for resolving latex errors, Developer \\
         \hline
    \end{tabular}
    \caption{Team Roles}
    \label{tab:team_roles}
\end{table}

\section{Workflow Plan} %S

The team will be using a main branch to store stable code. Every feature, bug fix or code refactor will require a new branch to be made from main and then a pull request would need to be created. Each pull request must be reviewed and approved by two other members of the team in order for it to be merged into the main branch. This will ensure a decrease in error prone code being added to the main branch. The pull request will be merged as a squash commit.

Issues will be discussed and created by all of the group members and will be stored in the GitHub issue tracker. Issues will be classified by three classes: feature, bug, and refactor. Each member will be assigned issues every week to work on when the implementation process has started. The team will establish team specific milestones, in order to keep track with deadlines and deliverables. 

\section{Proof of Concept Demonstration Plan}

What is the main risk, or risks, for the success of your project?  What will you
demonstrate during your proof of concept demonstration to convince yourself that
you will be able to overcome this risk?

\section{Technology}

\begin{itemize}
\item Specific programming language
\item Specific linter tool (if appropriate)
\item Specific unit testing framework
\item Investigation of code coverage measuring tools
\item Specific plans for Continuous Integration (CI), or an explanation that CI
  is not being done
\item Specific performance measuring tools (like Valgrind), if
  appropriate
\item Libraries you will likely be using?
\item Tools you will likely be using?
\end{itemize}

\section{Coding Standard}

\section{Project Scheduling}

\wss{How will the project be scheduled?}

\end{document}
